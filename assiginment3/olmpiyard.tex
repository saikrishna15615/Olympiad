\documentclass[12pt,-letter paper]{article}
\usepackage{gvv}
\begin{document}

\begin{center}                                                              \textbf{GEOMETRY}   
\end{center}

\begin{enumerate}

	\item Prove that the following assertion is true for  $n=3$  and  $n=5$,  and that it is false for every other natural number  $n>2 :$\\ 
		If $ a_{1}, a_{2}, ..., a_{n}$ are arbitrary real numbers, then $(a_{1} - a_{2})(a_{1} - a_{3})\cdots(a_{1} - a_{n}) + (a_{2} - a_{1})(a_{2} - a_{3})\cdots(a_{2} - a_{n})+\cdots+(a_{n} - a_{1})(a_{n} - a_{2})\cdots(a_{n} - a_{n-1}) \geq 0$.

\item Consider a convex polyhedron $P_{1}$ with nine vertices $A_{1}, A_{2}, \ldots, A_{9}$; \\
let $P_{i}$ be the polyhedron obtained from $P_{1}$ by a translation that moves vertex $A_{1}$ to $A_{i}$ ($i=2,3,\ldots,9$). Prove that at least two of the polyhedra $P_{1}, P_{2}, \ldots, P_{9}$ have an interior point in common.

\item Prove that for every natural number $m$, there exists a finite set $S$ of points in a plane with the following property: For every point $A$ in $S$, there are exactly $m$ points in $S$ which are at unit distance from $A$.


\item Prove that if $n \geq 4$, every quadrilateral that can be inscribed in a circle can be dissected into $n$ quadrilaterals each of which is inscribable in a circle.

	\item Given four distinct parallel planes, prove that there exists a regular tetrahedron with a vertex on each plane.

\item Point $O$ lies on line $g ; \myvec{OP_{1}}, \myvec {OP_{2}}, \ldots, \myvec {OP_{n}}$ are unit vectors such that points $P_{1}, P_{2}, \ldots, P_{n}$ all lie in a plane containing $g$ and on one side of $g$. Prove that if $n$ is odd, \begin{align}|\myvec{OP_{1}} + \myvec{OP_{2}} + \cdots + \myvec {OP_{n}}| \geq 1\end{align} Here $\myvec {OM}$ denotes the length of vector $\myvec {OM}$.
\item Determine whether or not there exists a finite set $M$ of points in space not lying in the same plane such that, for any two points $A$ and $B$ of $M$, one can select two other points $C$ and $D$ of $M$ so that lines $AB$ and $CD$ are parallel and not coincident.

\item A soldier needs to check on the presence of mines in a region having the shape of an equilateral triangle. The radius of action of his detector is equal to half the altitude of the triangle. The soldier leaves from one vertex of the triangle. What path shouid he follow in order to travel the least possible distance and still accomplish his mission?



\end{enumerate}

\begin{center}                  

	\textbf{NUMBER THEORY}                                            
\end{center}

\begin{enumerate}

\item Prove that the set of integers of the form $2^k - 3 (k=2,3,\ldots)$ contains an infinite subset in which every two members are relatively prime.

\item Prove that from a set of ten distinct two-digit numbers (in the decimal system), it is possible to select two disjoint subsets whose members have the same sum.

\item Let $m$ and $n$ be arbitrary non-negative integers. Prove that\begin{align} \frac{(2m)!(2n)!}{m!n!(m+n)!} \end {align} is an integer.$ (0! = 1.)$



\end{enumerate}

\begin{center}       

\textbf{ALGEBRA}                  
\end{center}             
		
\begin{enumerate}

\item All the faces of tetrahedron $ABCD$ are acute-angled triangles. We consider all closed polygonal paths of the form $XYZTX$ defined as follows: $X$ is a point on edge $AB$ distinct from $A$ and $B$; similarly, $Y, Z, T$ are interior points of edges $BC, CD, DA$, respectively. Prove:\\
\begin{enumerate}[label={$\brak{\Alph*}$}] \item If $\angle DAB + \angle BCD \neq \angle CDA + \angle ABC$, then among the polygonal paths, there is none of minimal length. \item If $\angle DAB + \angle BCD = \angle CDA + \angle ABC$, then there are infinitely
many shortest polygonal paths, their common length being $2AC \sin(\alpha/2)$, where $\alpha = \angle BAC + \angle CAD + \angle DAB$. \end {enumerate}

\item Let $A = (a_{ij}), i, j = 1, 2, \ldots, n$ be a square matrix whose elements are nonnegative integers. Suppose that whenever an element $a_{ij} = 0$, the sum of the elements in the $i$-th row and the $j$-th column is $> n$. Prove that the sum of all the elements of the matrix is $\geq n^2/2$.

\item Find all solutions $(x_{1}, x_{2}, x_{3}, x_{4}, x_{5})$ of the system of inequalities \\
\begin{align} (x^2_{1} - x_{3}x_{5})(x^2_{2} - x_{3}x_{5}) &\leq 0 \\ (x^2_{2} x_{4}x_{1})(x^2_{3} - x_{4}x_{1}) &\leq 0 \\ (x^2_{3} - x_{5}x_{2})(x^2_{4} x_{5}x_{2}) &\leq 0 \\ (x^2_{4} - x_{1}x_{3})(x^2_{5} - x_{1}x_{3}) &\leq 0 \\ (x^2_{5} x_{2}x_{4})(x^2_{1} - x_{2}x_{4}) &\leq 0
\end{align} where $x_{1}, x_{2}, x_{3}, x_{4}, x_{5}$ are positive real numbers.
\item Let $f$ and $g$ be real-valued functions defined for all real values of $x$ and $y$, and satisfying the equation \begin{align} f(x+y) + f(x-y) = 2f(x)g(y)\end{align} for all $x, y$. Prove that if $f(x)$ is not identically zero, and if $|f(x)| \leq 1$ for all $x$, then $|g(y)| \leq 1$ for all $y$.


\item Let $a$ and $b$ be real numbers for which the equation \begin{align}x^4 + ax^3 + bx^2 + ax + 1 = 0\end{align} has at least one real solution. For all such pairs $(a, b)$, find the minimum value of $a^2 + b^2$.

\item Let $G$ be a set of non-constant functions of the real variable $x$ of the form \begin{align}f(x) = ax + b,\end{align} where $a$ and $b$ are real numbers, and $G$ has the following properties: \begin{enumerate}[label={$\brak{\Alph*}$}] \item If $f$ and $g$ are in $G$, then $g \circ f$ is in $G$; here $(g \circ f)(x) = g[f(x)]$. \item If $f$ is in $G$, then its inverse $f^{-1}$ is in $G$; here the inverse of $f(x) = ax + b$ is $f^{-1}(x) = (x - b)/a$. \item For every $f$ in $G$, there exists a real number $x_{f}$ such that $f(x_{f}) = x_{f}$. Prove that there exists a real number $k$ such that $f(k) = k$ for all $f$ in $G$.

\item Let $a_{1}, a_{2}, \ldots, a_{n}$ be positive numbers, and let $q$ be a given real number such that $0 < q < 1$. Find $n$ numbers $b_{1}, b_{2}, \ldots, b_{n}$ for which \begin{enumerate}[label={$\brak{\Alph*}$}] \item $a_{k} < b_{k}$ for $k = 1, 2, \ldots, n$,\item $q < \dfrac{b_{k+1}}{b_{k}} < \dfrac{1}{q}$ for $k = 1, 2, \ldots, n - 1$, \item $b_{1} + b_{2} + \cdots + b_{n} < \dfrac{1+q}{1-q}(a_{1} + a_{2} + \cdots + a_{n})$.

\end{enumerate}
\end{document}
